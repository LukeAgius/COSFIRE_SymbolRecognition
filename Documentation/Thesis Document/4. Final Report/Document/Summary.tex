\chapter{Summary}

Handheld devices have become ubiquitous due to their flexibility and enhanced productivity. Such devices allow users to write freely on them or to manually sketch symbols. The automatic recognition of handwriting and hand-drawn sketches is important to convert the manual input to a digital representation. Examples of applications include recognition of handwritten letters, digits, signatures, musical notes, electrical and architectural symbols, among others. The objective of this dissertation is to investigate and determine the effectiveness of the COSFIRE filters against isolated symbol recognition with or without degradation applied to the images in question.\\

The approach used, in order to tackle this problem is based upon the work of G.Azzopardi and N.Petkov where a trainable filter, called COSFIRE, inspired by shape-selective neurons in area v4 of the visual cortex. Previously this filter was used successfully in the detection of vascular bifurcations, recognition of handwritten digits and detection and recognition of traffic signs. \\

A thorough literature review is presented which focuses on state-of-the-art symbol recognition methods and covers progress in symbol recognition. Several algorithms are presented along with an explanation on how they work. An introduction about COSFIRE filters is then given to put the reader in context. The COSFIRE filter is then described in detail. This includes the COSFIRE filters' configuration and application stages as well. Also a detailed description is given on how COSFIRE filters are used in the classifier built for this project.\\

Experimental results are then presented which reflect experiments executed against the first three categories of datasets, from the GREC'11 contests, which are publicly available. The following table lists the recognition rates achieved. Results are then interpreted in a discussion which will point out the strengths and weakness of the taken approach. Finally we draw conclusions regarding the effectiveness of COSFIRE filters in isolated symbol recognition. \\

The proposed approach is effective for the automatic recognition of architectural and electrical symbols. It is robust to scalability, noise and geometrical transformations. In most cases it outperforms the state of the art methods for the GREC datasets. In particular, the results that we achieve for the datasets characterised with different levels of degradation and noise are the best ever reported.